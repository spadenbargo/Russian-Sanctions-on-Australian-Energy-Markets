\section{Introduction}
\subsection{Background}

The present military action in Ukraine began on the 24th of February 2022 following the Russian Federation’s declaration of a “special military operation” in Ukraine \cite{intro1}. In response to this military action, the Russian Federation has been subject to an array of sanctions against individual Russians, Russian businesses, and Russia itself. Many of these sanctions directly target Russian oil and natural gas production, as well as the ability of Russia to sell this oil and natural gas to neighbours.
\medskip

Russia is a major entity in the global energy market. Russia is the third largest oil producer, accounting for ~12\% of global oil production, and is the second largest gas producer, accounting for ~17\% of natural gas production globally \cite{intro2}, thus, the impact of sanctions aimed at limiting the scale at which Russia is able to produce and export oil and gas is likely to have had a significant impact on the markets for these key commodities. Since the beginning of the war, the United States and Canada have banned Russian oil imports, whilst many other nations have specifically targeted sanctions and oil and gas producing companies in Russia. This is expected to both affect the global prices of oil and gas as well as the volatility of these prices, due to the rapidly changing nature of these sanctions.

\subsection{Aim}

This report will examine the impact of oil and gas related sanctions imposed upon the Russian Federation on the volatility of the price for crude oil and liquid natural gas (LNG) in both global and domestic (Australian) markets. We examine these impacts through the creation of a GARCH(1, 1) model, which was specified according to both relevant econometric literature and statistical tests. We then fit this model to the global prices of crude oil and liquid natural gas, and compare the realised level of volatility in these markets to a counterfactual generated by the specified model in the absence of sanctions, quantifying the expected magnitude of volatility in this forecast to the realised volatility in these markets, with sanctions in place. Additionally, we examine the possible impacts that sanctions on Russian gas exports may have on the price of electricity in Australia using a multivariate vector autoregressive model, and explore the possibility of a cointegrating relationship between global oil and gas spot prices to examine the possible impacts that sanctions on one commodity may have upon the other.
\medskip

Such an analysis is of interest to econometricians and policy makers alike as we aim to quantify the effect of a coordinated campaign of sanctions and economic warfare against a large oil and gas producing nation on the prices of oil and gas – critical commodities that impact the costs of producing and transporting a significant quantity of global goods and services, impacting all sectors of the global economy. In the Australian context, such a report is of interest as we examine the magnitude to which the Australian oil and gas markets are sensitive to sharp exogenous changes in the global prices of oil and gas.
