\section{Literature Review}
The sanctions imposed on Russia in response to its invasion of Ukraine have had an impact on gas and oil markets due to Russia’s position as a major producer of both commodities. Quantitative data is commonly used to forecast and explain changes in these oil and gas markets, including in response to shocks from sanctions on producing countries. In this review, the possible outcomes and ongoing effects of the current sanctions imposed on Russia will be explored through an analysis of previous papers that investigate the workings of these markets in response to sanctions and other shocks. These papers will be used to assess the relationship between sanctions and other shocks on the price and volatility of oil and gas. The findings from the studied papers will provide a helpful context and background understanding to the statistical modeling to be undertaken to assess the unique impact which the recent Russian sanctions are having upon these markets.
\medskip

Brown \cite{aid1} analysed the effects of the United States’ sanctions imposed on Iran, Russia, and Venezuela, as of 2020, and found them all to have some effect on oil markets. Brown found the effect that the sanctions had on global oil markets was dependent on the target and thoroughness of the sanctions. Another important factor was the design elements in the sanctions specifically designed to avoid global impacts by certifying that oil markets outside of the sanctioned country were adequately supplied. In the examples discussed in this paper, these included coordinating with oil-producing nations, significant reduction exceptions for industries defined as critical, and ‘wind-down’ periods after the imposition of the sanctions. While these approaches attempted to reduce the global impact on oil markets, in the case of the Iranian and Venezuelan sanctions, which did cause an immediate drop in oil production, the global price of oil did rise. However, likely due to these design elements, as Brown hypothesises, the price soon adjusted as other oil-producing nations increased production. While this paper does analyse the effects of sanctions on global oil markets, there is little empirical data analysing this effect, thus it is hard to understand to what amount these sanctions had an effect, or to apply the data to any past or future sanction events. In the case of the recent sanctions on Russia, which have affected Russian oil production and its ability to export more than the sanctions which started in 2014, this paper gives little insight into the effect of these sanctions, considering the greater amount of global oil production which is being impacted by them. Due to the rapid nature of Russia’s recent sanctions, this article would imply a larger effect, as there was not enough time to ensure adequate supply and design the sanctions so as not to have such a large effect. While this article offers little insight into the empirical relationship between oil markets and sanctions, it does establish a strong relationship between them, informing the research in this paper.
\medskip

In their article, Mohaddes and Pesaran \cite{aid2} try to quantitatively assess the impact of country-specific oil shocks on the global economy using a counterfactual analysis. They have used various vector autoregressive (VAR) models to find the relationships between price shocks in one country and the global oil price. By specifying the model to a Global-VAR model, the findings are able to specifically analyse the effects of a country-specific shock, thus considering the actions of other oil-producing nations in response. This paper is then able to deliver specific empirical insights into the effects of a country’s shock, such as the recent Russian sanctions, on the global oil market. They find that when the affected country produces large amounts of oil, there is far less spare capacity in other producing nations, and the oil markets can be affected. The example of Saudi Arabia is used, the largest oil-producing nation at the time of writing (and still a very significant one \cite{aid3}). Using their model, with an 11\% long-run reduction in oil production, it results in a price increase of 22\% in oil. Russia’s production from February (pre-invasion) to March appears to have dropped by a fiftieth of that \cite{aid4}, but utilising the approach of this paper, the impact of Russia’s sanctions can be isolated within its effect on the global markets. Other approaches to control for confounding variables include the synthetic control method, discussed by Abadie, Diamond, and Hainmueller \cite{aid5}, in which an aggregate of other indicators are used to simulate the affected country.
\medskip

Both of these papers provide justification for the research question this report poses, and suggest there should be some effect on global oil and gas prices due to the imposition of sanctions that target their ability to produce and export these commodities. The econometric analysis in Mohaddes and Peseran’s paper provides evidence of a VAR model being appropriate when conducting a counterfactual analysis, however, as discussed in their paper, the VAR model has limitations. It cannot account for as many time series data as a GARCH model and is not able to accurately model data for both oil and gas data which is the goal this report. The GARCH(1,1) used to conduct the analysis in this report is based upon this research as well as other findings and patterns identified through data inspection into the prices of these commodities.
